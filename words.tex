additive synthesis- is a sound synthesis technique based on adding sine waves together\ref{sec:waveMethods}
analog synthesizer-is a synthesizer that uses analog circuits and analog signals to generate sound electronically
attack-period for which sound needs to reach its peak from zero
beat-rhythmic movement
big-endian- the most significant byte (the "big end") of the data is placed at the byte with the lowest address
decay-period when the sound level decreases to sustain level
flooring-function that takes as input a real number and gives as output the greatest integer less than or equal to
frequency-the number of waves that pass a fixed point in unit time
harmonics-is a signal or wave whose frequency is an integral (whole-number) multiple of the frequency of some
midi-is a protocol designed for recording and playing back music on digital synthesizers- is a synthesizer that uses digital signal processing techniques to make musical sounds
normalization-process of bringing or returning something to a normal condition or state.
phase-shifted - is the angle that the waveform has shifted from a certain reference point along the horizontal zero axis
release-period until the sound fades to silence
subtractive synthesis-is a sound synthesis technique based on subtracting sine waves together
sustain-period when the sound doesn't change until the key is realised 
tempo-the speed at which a passage of music is or should be played
wav-is an audio file format standard, developed by Microsoft and IBM
waveform-a usually graphic representation of the shape of a wave that indicates its characteristics (such as frequency and amplitude)

\label{sec:Blah}








----from page 10
tickdiv intervals -> ????
chunk - piece of data encoded into a MIDI file. There are two different types of chunks, the header chunk () and the track chunk (\ref{gloss:track_chunk}).  
Meta events - events occur within MIDI tracks and specify various kinds of information and actions. They may appear at any time within the track. 
wavetable - a table of stored sound waves that are digitized samples of actual recorded sound
RIFF - a repeated chord progression or refrain in music; it is a pattern, or melody, often played by the rhythm section instruments or solo instrument, that forms the basis or accompaniment of a musical composition

channels - One audio recording made on a portion of the width of a multitrack tape; a single path that an audio signal travels or can travel through a device from an input to an output

tag -> ???
World - an abstract type representing the environment, or the state of the world
little-endian - The least significant byte (LSB) value is at the lowest address. The other bytes follow in increasing order of significance. This is akin to right-to-left reading in hexadecimal order.
endianness - refers to the order of bytes (or sometimes bits) within a binary representation of a number. It can also be used more generally to refer to the internal ordering of any representation, such as the digits in a numeral system or the sections of a date.
just intonation - the tuning of musical intervals as whole number ratios of frequencies. 
digital synthesizer - a synthesizer (electronic musical instrument that generates audio signals) that uses digital signal processing techniques to make musical sounds. 
